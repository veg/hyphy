\documentclass[11pt,letterpaper]{article}
\usepackage[utf8]{inputenc}
\usepackage{amsmath}
\usepackage{amsfonts}
\usepackage{amssymb}
\usepackage{cite}
\usepackage[margin=1in]{geometry}
\usepackage{hyperref}

\title{Supplementary Material: A Technical Review of Methodologies for Estimating the Probability of Site-Level Conservation}
\author{Sergei L. Kosakovsky Pond \and Darren P. Martin}
\date{\today}

\begin{document}

\maketitle

\section{Introduction}
The quantification of site-specific conservation is a cornerstone of comparative genomics, yet the statistical machinery underlying various tools differs significantly in its assumptions regarding neutral drift and rate heterogeneity. To provide a rigorous context for the development of 	B-STILL, we dissect the technical mechanisms by which existing methods estimate the probability of conservation. Ignore the nuances of these implementations at your own peril, as each introduces distinct statistical trade-offs between sensitivity and specificity.

\section{Methodological Dissections}

\subsection{phyloP: Phylogenetic P-values}
The 	phyloP tool, part of the PHAST suite, represents perhaps the most direct application of likelihood ratio tests (LRT) to site-level conservation \cite{pollard2010detection}.
\begin{itemize}
    \item 	Estimation Logic: For each site, 	phyloP evaluates the observed number of substitutions against a pre-calibrated neutral phylogenetic tree. The null hypothesis assumes that substitutions follow a neutral Markov process (typically calibrated using four-fold degenerate sites).
    \item 	Quantification of Invariance: For a strictly invariant site, the surprisingness is a direct function of the total neutral tree length ($T$). Under a Poisson approximation, the probability of zero substitutions is $e^{-T}$. 	phyloP outputs a score defined as $-\log_{10}(p	ext{-value})$, where $p$ is the probability of the data under the neutral model.
    \item 	Limitation: As a site-independent method, it lacks the ability to share information across the gene, making it highly sensitive to the accuracy of the provided neutral tree.
\end{itemize}

\subsection{GERP++: Genomic Evolutionary Rate Profiling}
	GERP++ eschews p-values in favor of a count-based metric known as 	Rejected Substitutions (RS) \cite{davydov2010hierarchical}.
\begin{itemize}
    \item 	Estimation Logic: The method calculates the expected number of substitutions ($E$) that would have occurred at a site under neutral evolution given the tree topology and branch lengths. It then subtracts the observed number of substitutions ($O$).
    \item 	Quantification of Invariance: For an invariant site ($O=0$), the RS score is exactly equal to the neutral tree length ($RS = E$). This provides an intuitive, additive measure of constraint: an RS score of 5.0 implies that evolution has successfully ``rejected'' five expected mutations.
    \item 	Limitation: Like 	phyloP, 	GERP++ treats sites in isolation, failing to account for the gene-wide distribution of selective pressures.
\end{itemize}

\subsection{phastCons: Spatial Context via HMMs}
Unlike independent site methods, 	phastCons utilizes a 	Phylogenetic Hidden Markov Model (phylo-HMM) to contextualize conservation \cite{siepel2005evolutionarily}.
\begin{itemize}
    \item 	Estimation Logic: The model toggles between two hidden states: ``neutral'' and ``conserved.'' The state transitions are governed by a set of parameters that reflect the expected length of conserved elements.
    \item 	Quantification of Invariance: It assigns a posterior probability (from 0 to 1) that a site belongs to the conserved state. A strictly invariant site flanked by high variation will receive a lower probability than one residing within a cluster of conserved positions.
    \item 	Limitation: While effective for discovering conserved motifs, the smoothing effect of the HMM can obscure sharp, site-specific signatures of purifying selection.
\end{itemize}

\subsection{Rate4Site: Empirical Bayesian Rate Inference}
For amino acid alignments, 	Rate4Site is the established gold standard for quantifying protein-level conservation \cite{pupko2002rate4site}.
\begin{itemize}
    \item 	Estimation Logic: The method uses Empirical Bayesian inference to assign a relative evolutionary rate to every site, incorporating a substitution matrix (e.g., WAG or JTT) and the phylogeny.
    \item 	Quantification of Invariance: An invariant site yields a posterior probability distribution sharply peaked at the lowest possible rate. This allows for a continuous ranking of sites even when multiple sites appear identical at the sequence level.
    \item 	Limitation: It typically assumes a single global rate for the entire protein, which is fundamentally insufficient for capturing the $dN/dS$ dynamics essential for codon-level analysis.
\end{itemize}

\subsection{Mechanistic Codon Models (HyPhy / PAML)}
Codon-based models provide a more nuanced view of selection by estimating the ratio of non-synonymous to synonymous substitutions ($\omega = \beta/\alpha$) \cite{kosakovsky2005hyphy, yang2007paml}.
\begin{itemize}
    \item 	Estimation Logic: These models use synonymous drift ($\alpha$) as an internal control. Conservation is inferred when $\beta < \alpha$.
    \item 	Quantification of Invariance: Methods like FEL or FUBAR estimate the posterior probability $P(\beta < \alpha | 	ext{Data})$. However, at strictly invariant sites where $\alpha = \beta = 0$, the standard selection test is often statistically uninformative.
    \item 	Conceptual Gap: This methodological failure at the limit of zero substitutions is the primary motivation for the high-resolution grid and EBF approach implemented in 	B-STILL.
\end{itemize}

\subsection{EVE: Generative Deep Learning}
The EVE (Evolutionary model of Variant Effect) method utilizes Variational Autoencoders (VAEs) trained on deep sequence alignments \cite{frazer2021disease}.
\begin{itemize}
    \item Estimation Logic: EVE learns the latent multi-dimensional distribution of sequences. It quantifies conservation through the ``unlikeliness'' of a mutation within the learned manifold.
    \item Quantification of Invariance: For invariant sites, the model's latent space dictates that any amino acid other than the consensus has an extremely low probability.
    \item Limitation: While powerful, these models require large-scale alignments and are often computationally prohibitive compared to grid-based Bayesian methods.
\end{itemize}

\section{Standardizing the Neutral Baseline for Benchmarking}
To rigorously compare B-STILL against site-independent tools like phyloP and GERP++, it is essential to establish a conceptually equivalent neutral baseline. Traditionally, these tools rely on neutral trees calibrated using four-fold degenerate sites---an approximation that is fundamentally insufficient as it discards the majority of the synonymous signal and introduces potential compositional biases.

We propose a Codon-derived Neutrality protocol for benchmarking:
\begin{enumerate}
    \item Synonymous Tree Estimation: Utilize a mechanistic codon model (e.g., MG94) to estimate the synonymous substitution rate ($\alpha$) across the entire alignment. This leverages every synonymous substitution opportunity, yielding a significantly more stable and data-dense phylogeny than 4-fold site counting.
    \item Neutral Reference Extraction: The true neutral baseline for a coding region is defined as the tree where branch lengths ($L$) represent expected synonymous change ($L_{neutral} = \alpha \times T$). This tree represents the expected drift at a site where $\beta = \alpha$.
    \item Cross-Tool Calibration: Export this synonymous tree and the associated GTR substitution parameters into a PHAST-compatible \texttt{.mod} file. This ensures that the ``surprise'' measured by phyloP is statistically comparable to the constraint inferred by B-STILL, moving the comparison from one of data-proxies to one of statistical inference frameworks.
\end{enumerate}

\bibliographystyle{plain}
\bibliography{references}

\end{document}
